\documentclass{beamer}
\usetheme{CambridgeUS}

\setbeamertemplate{caption}[numbered]{}

\usepackage{enumitem}
\usepackage{amsmath}
\usepackage{amssymb}
\usepackage{gensymb}
\usepackage{graphicx}
\usepackage{txfonts}

\def\inputGnumericTable{}

\usepackage[latin1]{inputenc}                                 
\usepackage{color}                                            
\usepackage{array}                                            
\usepackage{longtable}                                        
\usepackage{calc}                                             
\usepackage{multirow}                                         
\usepackage{hhline}                                           
\usepackage{ifthen}
\usepackage{caption}

\title{AI1110 \\ Assignment 1}
\author{Ritvik Sai C \\ CS21BTECH11054}
\date{}
\begin{document}
	% The title page
	\begin{frame}
		\titlepage
	\end{frame}
	
	% The table of contents
	\begin{frame}{Outline}
    		\tableofcontents
	\end{frame}
	
	% The question
	\section{Question}
	\begin{frame}{Question 7.a}
A page from a saving bank account passbook is given above:\\

\end{frame}

\begin{frame}
\begin{table}[ht]
\begin{tabular}{|c|c|c|c|c|}
\hline
Date & Particulars & Amount Withdrawn & Amount Deposited & Balance\\
\hline
Jan 7,2016 & B/F & & & 3000.00\\
\hline
Jan 10,2016 & By Cheque & & 2600.00 & 5600.00\\
\hline
Feb 8,2016 & To Self & 1500.00 & & 4100.00\\
\hline
Apr 6,2016 & By Cheque & 1200.00 & & 2000.00\\
\hline
May 4,2016 & By Cash & & 6500.00 & 8500.00\\
\hline
May 27,2016 & By Cheque & & 1500.00 & 10000.00\\
\hline
\end{tabular}
\end{table}
\end{frame}

\begin{frame}

Calculate the interest for the 6 months from January to June 2016, at 6 percent per annum\\

\end{frame}

\begin{frame}
 If the account is closed on 1st July 2016, find the amount received by the account holder\\

\end{frame}
	
	% The solution
	\section{Solution}
	\begin{frame}{Solution}
	

From Jan 7 to Jan 10 3000.00 rupees was in the  bank for 3 days $\implies 3000\times3\times6/(100\times365)=1.48$ rupees\\
From Jan 10 to Feb 8 5600.00 rupees was in the  bank for 29 days $\implies 5600\times29\times6/(100\times365)=26.70$ rupees \\
From Feb 8 to Apr 6 4100.00 rupees was in the  bank for 58 days $\implies 4100\times58\times6/(100\times365)=39.09$ rupees\\ 
From Apr 6 to May 4 2000.00 rupees was in the  bank for 28 days $\implies 2000\times28\times6/(100\times365)=9.21$ rupees\\
From May 4 to May 27 8500.00 rupees was in the  bank for 23 days $\implies 8500\times23\times6/(100\times365)=32.14$ rupees\\
From May 27 to June 30 10000.00 rupees was in the  bank for 34 days $\implies 10000\times34\times6/(100\times365)=55.89$ rupees\\
$\therefore$ Total Interest is 164.51 rupees\\
\end{frame}

\begin{frame}
 If the account is closed on July 1st then the amount received by the account holder is $ 10000 + 164.51 =10164.51$ rupees \\



\end{frame}
	
\end{document}\textsl{•}